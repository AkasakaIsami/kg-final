\section{方法}
基于上述研究目标,图~\ref{fig:研究方法}给出了研究方法概述,主要将方法定义为三个步骤:

\begin{enumerate}
    \item 语句生成。基于已有的知识图谱生成语句集,包括同时由事实语句与反事实语句构成、且数据量较大的训练集,以及仅包括反事实语句的验证集。
    \item 模型训练。以生成的训练集为输入,对预训练的语言模型进行训练。
    \item 模型验证。以生成的验证集为输入,对训练好的语言模型进行验证,并对输出结果进行评估。
\end{enumerate}

\begin{figure}[htb]
    \centering
    \includegraphics[width=0.5\textwidth]{images/研究方法.png}
    \caption[研究方法]{研究方法}
    \label{fig:研究方法}
\end{figure}

基于上述研究方法与具体步骤,本文将研究问题定义为三个部分,包括知识图谱选择、基于图谱的事实与反事实语句生成、以及语言模型选择。
本节将针对上述定义的三个研究问题进行说明。

\subsection{知识图谱选择}
由于开源知识图谱的选择众多且具有多样性,表现在知识领域、图谱规模、图谱文件类型以及数据结构等多个方面,因此选择合适的知识图谱用于事实与反实施语句生成是本文的第一个研究问题。
合适的知识图谱应包含以下几个特点:
\begin{enumerate}
    \item 以常识为知识领域,能够以常识为事实提供事实知识。
    \item 具有一定规模,能够提供支撑模型训练数据量的数据。
    \item 图谱内数据结构合理,且最好能为事实生成提供有效的额外信息。
\end{enumerate}

针对这三个特点,本文选择ConceptNet\footnote{ConceptNet:https://conceptnet.io/}作为实验知识图谱。
ConceptNet是一个免费提供的语义网络,同时也是一个知识图谱,旨在帮助计算机理解人们使用的词语的含义。
ConceptNet起源于众包项目Open Mind Common Sense,该项目于1999年在麻省理工学院媒体实验室启动。
该项目已经发展到包括来自其他众包资源的知识、专家创造的资源和有目的的游戏。
ConceptNet5版本已经包含有2800万关系描述。

相同于常规开源知识图谱,ConceptNet的原始数据集格式为CSV文件,但与此同时,ConceptNet还提供REST API 以获取 JSON-LD 格式的数据\footnote{ConceptNet web API:https://github.com/commonsense/conceptnet5/wiki/API}。
以下代码给出了使用REST API获取以实体$example$为主语的一组三元组的数据后得到的返回结果样例,同时展示了ConceptNet的数据结构。


\begin{lstlisting}[language=json,firstnumber=1]
{
    "@id": "/a/[/r/IsA/,/c/en/example/n/,/c/en/information/n/]",
    "dataset": "/d/wordnet/3.1",
    "end": {
        "@id": "/c/en/information/n",
        "label": "information",
        "language": "en",
        "sense_label": "n",
        "term": "/c/en/information"
    },
    "license": "cc:by/4.0",
    "rel": {
        "@id": "/r/IsA",
        "label": "IsA"
    },
    "sources": [
        {
            "@id": "/s/resource/wordnet/rdf/3.1",
            "contributor": "/s/resource/wordnet/rdf/3.1"
        }
    ],
    "start": {
        "@id": "/c/en/example/n",
        "label": "example",
        "language": "en",
        "sense_label": "n",
        "term": "/c/en/example"
    },
    "surfaceText": "[[example]] is a type of [[information]]",
    "weight": 2.0
}
\end{lstlisting}

可以观察到,返回的JSON数据中除去构成三元组的包含4-10行、22-28行的实体信息以及12-15行的关系信息外,同时还提供的$surfaceText$与$weight$信息。
一些ConceptNet的数据是从自然语言文本中提取的,$surfaceText$的值展示提取这个数据的原始文本是什么,而$weight$值显示这个信息有多可信,当信息来自更多的来源或更可靠的来源时,$weight$值会更高。


\subsection{基于图谱的事实与反事实语句生成}


\subsection{语言模型选择}

