\section{方法}
基于上述研究目标,图~\ref{fig:研究方法}给出了研究方法概述,主要将方法定义为三个步骤:

\begin{enumerate}
    \item 语句生成。基于已有的知识图谱生成语句集,包括同时由事实语句与反事实语句构成、且数据量较大的训练集,以及仅包括反事实语句的验证集。
    \item 模型训练。以生成的训练集为输入,对预训练的语言模型进行训练。
    \item 模型验证。以生成的验证集为输入,对训练好的语言模型进行验证,并对输出结果进行评估。
\end{enumerate}

\begin{figure}[htb]
    \centering
    \includegraphics[width=0.5\textwidth]{images/研究方法.png}
    \caption[研究方法]{研究方法}
    \label{fig:研究方法}
\end{figure}

基于上述研究方法与具体步骤,本文将研究问题定义为三个部分,包括知识图谱选择、基于图谱的事实与反事实语句生成、以及语言模型选择。
本节将针对上述定义的三个研究问题进行说明。

\subsection{知识图谱选择}
由于网络上开源知识图谱的选择众多且具有多样性,表现为图谱规模、图谱文件结构、知识领域以及数据结构等,因此选择合适的知识图谱用于事实与反实施语句生成是本文的第一个研究问题。
为简化语句生成,选择的知识图谱应包含一下几个特点:
\begin{enumerate}
    \item 具有一定规模,
\end{enumerate}

ConceptNet是一个免费提供的语义网络(semantic network),旨在帮助计算机理解人们使用的词语的含义。
ConceptNet起源于众包项目Open Mind Common Sense,该项目于1999年在麻省理工学院媒体实验室启动。
自那以后,该项目已经发展到包括来自其他众包资源的知识、专家创造的资源和有目的的游戏

\subsection{基于图谱的事实与反事实语句生成}


\subsection{语言模型选择}

